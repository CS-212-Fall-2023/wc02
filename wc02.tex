\documentclass[a4paper]{exam}

\usepackage{amsmath,amssymb, amsthm}
\usepackage{geometry}
\usepackage{graphicx}
\usepackage{hyperref}

\title{Weekly Challenge 02: Deterministic Finite Automata (DFA)}
\author{CS 212 Nature of Computation\\Habib University}
\date{Fall 2023}

\qformat{{\large\bf \thequestion. \thequestiontitle}\hfill}
\boxedpoints

\printanswers

\begin{document}
\maketitle

\begin{questions}
  
\titledquestion{The Complement Language}

  Consider the following finite automata and their languages.
  \begin{itemize}
  \item $M_1=(Q, \{0,1\}, \delta, q_o, F)$ and its language, $L_1=L(M_1)$, and
  \item $M_2=(Q, \{0,1\}, \delta, q_o, Q-F)$ and its language, $L_2=L(M_2)$.
  \end{itemize}

  Prove or disprove the following claim,
  
  \centerline{\fbox{
      $L_1 = L_2'$
    }
  }
  where $L'$ is the set-complement of $L$.

  Uncomment below to enter your solution.
  \begin{solution}

    \begin{proof}  We prove that $L_1 \subseteq L_2'$ and that $L_2' \subseteq L_1$. We leverage the observation that because the set of states, alphabet, and transition function are identical in $M_1$ and $M_2$, both machines are in the same state after reading a given string.
      
      \underline{Case 1}: $L_1 \subseteq L_2'$

      Assume a string $w\in L_1$ and let $q$ be the state of $M_1$ after reading $w$.\\
      Then, $q\in F$.\\
      $\therefore q\not\in (Q-F)$\\
      $\therefore w\not\in L_2$\\
      $\therefore w\in L_2'$\\\smallskip

      \underline{Case 2}: $L_2' \subseteq L_1$

      Assume a string $w\in L_2'$ and let $q$ be the state of $M_2$ after reading $w$.\\
      Then, $q\not\in (Q-F)$.\\
      $\therefore q\in F$\\
      $\therefore w\in L_1$      
    \end{proof}
  \end{solution}
  
\end{questions}
\end{document}

%%% Local Variables:
%%% mode: latex
%%% TeX-master: t
%%% End:
