\documentclass[a4paper]{exam}

\usepackage{amsmath}
\usepackage{geometry}
\usepackage{graphicx}
\usepackage{hyperref}

\printanswers

\title{Weekly Challenge 02: Equivalence of Finite Automata}
\author{CS 212 Nature of Computation\\Habib University}
\date{Fall 2022}

\qformat{{\large\bf \thequestion. \thequestiontitle}\hfill}
\boxedpoints

\begin{document}
\maketitle

\begin{questions}
  
\titledquestion{NFA-DFA Equivalence}

  Theorem 1.39 in our textbook states that, ``Every nondeterministic finite automaton has an equivalent deterministic finite automaton'', and then provides a proof by construction.

  Prove that the DFA obtained from an NFA by applying the given construction is indeed equivalent. That is, show that the constructed DFA accepts the same language as the given NFA and vice versa.
  
  You are expected to submit an original proof (i.e. developed by you) that is correct and exhibits sound and precise argumentation. If you consult any sources for guidance, make sure to cite and acknowledge them duly.
  
  \begin{solution}
    % Enter your solution here.
  \end{solution}
\end{questions}
\end{document}

%%% Local Variables:
%%% mode: latex
%%% TeX-master: t
%%% End:
